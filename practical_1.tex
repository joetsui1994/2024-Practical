% Options for packages loaded elsewhere
\PassOptionsToPackage{unicode}{hyperref}
\PassOptionsToPackage{hyphens}{url}
%
\documentclass[
  11pt,
]{article}
\usepackage{amsmath,amssymb}
\usepackage{lmodern}
\usepackage{iftex}
\ifPDFTeX
  \usepackage[T1]{fontenc}
  \usepackage[utf8]{inputenc}
  \usepackage{textcomp} % provide euro and other symbols
\else % if luatex or xetex
  \usepackage{unicode-math}
  \defaultfontfeatures{Scale=MatchLowercase}
  \defaultfontfeatures[\rmfamily]{Ligatures=TeX,Scale=1}
\fi
% Use upquote if available, for straight quotes in verbatim environments
\IfFileExists{upquote.sty}{\usepackage{upquote}}{}
\IfFileExists{microtype.sty}{% use microtype if available
  \usepackage[]{microtype}
  \UseMicrotypeSet[protrusion]{basicmath} % disable protrusion for tt fonts
}{}
\makeatletter
\@ifundefined{KOMAClassName}{% if non-KOMA class
  \IfFileExists{parskip.sty}{%
    \usepackage{parskip}
  }{% else
    \setlength{\parindent}{0pt}
    \setlength{\parskip}{6pt plus 2pt minus 1pt}}
}{% if KOMA class
  \KOMAoptions{parskip=half}}
\makeatother
\usepackage{xcolor}
\usepackage[margin=1in]{geometry}
\usepackage{color}
\usepackage{fancyvrb}
\newcommand{\VerbBar}{|}
\newcommand{\VERB}{\Verb[commandchars=\\\{\}]}
\DefineVerbatimEnvironment{Highlighting}{Verbatim}{commandchars=\\\{\}}
% Add ',fontsize=\small' for more characters per line
\usepackage{framed}
\definecolor{shadecolor}{RGB}{248,248,248}
\newenvironment{Shaded}{\begin{snugshade}}{\end{snugshade}}
\newcommand{\AlertTok}[1]{\textcolor[rgb]{0.94,0.16,0.16}{#1}}
\newcommand{\AnnotationTok}[1]{\textcolor[rgb]{0.56,0.35,0.01}{\textbf{\textit{#1}}}}
\newcommand{\AttributeTok}[1]{\textcolor[rgb]{0.77,0.63,0.00}{#1}}
\newcommand{\BaseNTok}[1]{\textcolor[rgb]{0.00,0.00,0.81}{#1}}
\newcommand{\BuiltInTok}[1]{#1}
\newcommand{\CharTok}[1]{\textcolor[rgb]{0.31,0.60,0.02}{#1}}
\newcommand{\CommentTok}[1]{\textcolor[rgb]{0.56,0.35,0.01}{\textit{#1}}}
\newcommand{\CommentVarTok}[1]{\textcolor[rgb]{0.56,0.35,0.01}{\textbf{\textit{#1}}}}
\newcommand{\ConstantTok}[1]{\textcolor[rgb]{0.00,0.00,0.00}{#1}}
\newcommand{\ControlFlowTok}[1]{\textcolor[rgb]{0.13,0.29,0.53}{\textbf{#1}}}
\newcommand{\DataTypeTok}[1]{\textcolor[rgb]{0.13,0.29,0.53}{#1}}
\newcommand{\DecValTok}[1]{\textcolor[rgb]{0.00,0.00,0.81}{#1}}
\newcommand{\DocumentationTok}[1]{\textcolor[rgb]{0.56,0.35,0.01}{\textbf{\textit{#1}}}}
\newcommand{\ErrorTok}[1]{\textcolor[rgb]{0.64,0.00,0.00}{\textbf{#1}}}
\newcommand{\ExtensionTok}[1]{#1}
\newcommand{\FloatTok}[1]{\textcolor[rgb]{0.00,0.00,0.81}{#1}}
\newcommand{\FunctionTok}[1]{\textcolor[rgb]{0.00,0.00,0.00}{#1}}
\newcommand{\ImportTok}[1]{#1}
\newcommand{\InformationTok}[1]{\textcolor[rgb]{0.56,0.35,0.01}{\textbf{\textit{#1}}}}
\newcommand{\KeywordTok}[1]{\textcolor[rgb]{0.13,0.29,0.53}{\textbf{#1}}}
\newcommand{\NormalTok}[1]{#1}
\newcommand{\OperatorTok}[1]{\textcolor[rgb]{0.81,0.36,0.00}{\textbf{#1}}}
\newcommand{\OtherTok}[1]{\textcolor[rgb]{0.56,0.35,0.01}{#1}}
\newcommand{\PreprocessorTok}[1]{\textcolor[rgb]{0.56,0.35,0.01}{\textit{#1}}}
\newcommand{\RegionMarkerTok}[1]{#1}
\newcommand{\SpecialCharTok}[1]{\textcolor[rgb]{0.00,0.00,0.00}{#1}}
\newcommand{\SpecialStringTok}[1]{\textcolor[rgb]{0.31,0.60,0.02}{#1}}
\newcommand{\StringTok}[1]{\textcolor[rgb]{0.31,0.60,0.02}{#1}}
\newcommand{\VariableTok}[1]{\textcolor[rgb]{0.00,0.00,0.00}{#1}}
\newcommand{\VerbatimStringTok}[1]{\textcolor[rgb]{0.31,0.60,0.02}{#1}}
\newcommand{\WarningTok}[1]{\textcolor[rgb]{0.56,0.35,0.01}{\textbf{\textit{#1}}}}
\usepackage{graphicx}
\makeatletter
\def\maxwidth{\ifdim\Gin@nat@width>\linewidth\linewidth\else\Gin@nat@width\fi}
\def\maxheight{\ifdim\Gin@nat@height>\textheight\textheight\else\Gin@nat@height\fi}
\makeatother
% Scale images if necessary, so that they will not overflow the page
% margins by default, and it is still possible to overwrite the defaults
% using explicit options in \includegraphics[width, height, ...]{}
\setkeys{Gin}{width=\maxwidth,height=\maxheight,keepaspectratio}
% Set default figure placement to htbp
\makeatletter
\def\fps@figure{htbp}
\makeatother
\setlength{\emergencystretch}{3em} % prevent overfull lines
\providecommand{\tightlist}{%
  \setlength{\itemsep}{0pt}\setlength{\parskip}{0pt}}
\setcounter{secnumdepth}{-\maxdimen} % remove section numbering
\ifLuaTeX
  \usepackage{selnolig}  % disable illegal ligatures
\fi
\IfFileExists{bookmark.sty}{\usepackage{bookmark}}{\usepackage{hyperref}}
\IfFileExists{xurl.sty}{\usepackage{xurl}}{} % add URL line breaks if available
\urlstyle{same} % disable monospaced font for URLs
\hypersetup{
  pdftitle={Practical I: epidemiological surveillance and modelling (non-spatial)},
  hidelinks,
  pdfcreator={LaTeX via pandoc}}

\title{Practical I: epidemiological surveillance and modelling
(non-spatial)}
\author{}
\date{\vspace{-2.5em}26/27th January}

\begin{document}
\maketitle

In this first practical the student is expected to learn about
epidemiological case count data and standard techniques to visualise and
analyse them. Further, the student is expected to reflect on the issues
these data present and provide a perspective on how to improve disease
surveillance in order to better estimate epidemiological parameters in
the future.

The practical will start with a short 15-20 minute presentation by the
course convenor and demonstrators. The next 2h will be spent answering
predefined questions before a short wrap up and reflection at the end.
There will be a 15 minute break after ca. 1h ½.

\hypertarget{practical-i-questions}{%
\subsection{Practical I Questions}\label{practical-i-questions}}

\hypertarget{q1-how-is-public-health-surveillance-for-infectious-diseases-defined-ca.-5-minutes}{%
\paragraph{Q1: How is public health surveillance for infectious diseases
defined? (ca. 5
minutes)}\label{q1-how-is-public-health-surveillance-for-infectious-diseases-defined-ca.-5-minutes}}

~

\hypertarget{q2-what-are-common-disease-surveillance-data-and-what-are-their-pros-and-cons-name-4-at-most.-ca.-15-minutes}{%
\paragraph{Q2: What are common disease surveillance data and what are
their pros and cons? Name 4 at most. (ca. 15
minutes)}\label{q2-what-are-common-disease-surveillance-data-and-what-are-their-pros-and-cons-name-4-at-most.-ca.-15-minutes}}

~

\hypertarget{q3-what-are-some-of-the-common-challenges-with-public-health-surveillance-at-the-beginning-of-disease-outbreaks-ca.-5-minutes}{%
\paragraph{Q3: What are some of the common challenges with public health
surveillance at the beginning of disease outbreaks? (ca. 5
minutes)}\label{q3-what-are-some-of-the-common-challenges-with-public-health-surveillance-at-the-beginning-of-disease-outbreaks-ca.-5-minutes}}

~

\hypertarget{q4-describe-the-key-data-sources-necessary-to-estimate-the-time-varying-reproduction-number-rt-httpsacademic.oup.comajearticle1789150589262-focus-on-main-manuscript-and-high-level-insights.ca.-20-minutes}{%
\paragraph{\texorpdfstring{Q4: Describe the key data sources necessary
to estimate the time-varying reproduction number Rt:
\url{https://academic.oup.com/aje/article/178/9/1505/89262}? Focus on
main manuscript and high level insights.(ca. 20
minutes)}{Q4: Describe the key data sources necessary to estimate the time-varying reproduction number Rt: https://academic.oup.com/aje/article/178/9/1505/89262? Focus on main manuscript and high level insights.(ca. 20 minutes)}}\label{q4-describe-the-key-data-sources-necessary-to-estimate-the-time-varying-reproduction-number-rt-httpsacademic.oup.comajearticle1789150589262-focus-on-main-manuscript-and-high-level-insights.ca.-20-minutes}}

~

\hypertarget{q5-please-install-the-r-package-epiestim-and-load-it-httpscran.r-project.orgwebpackagesepiestimindex.html-and-familiarize-yourself-with-the-documentation.-remove-any-objects-from-your-workspace.-copy-the-code-to-install-the-package-below.-ca.-15-min}{%
\paragraph{\texorpdfstring{Q5: Please install the R-package `EpiEstim'
and load it
(\url{https://cran.r-project.org/web/packages/EpiEstim/index.html}), and
familiarize yourself with the documentation. Remove any objects from
your workspace. Copy the code to install the package below. (ca. 15
min)}{Q5: Please install the R-package `EpiEstim' and load it (https://cran.r-project.org/web/packages/EpiEstim/index.html), and familiarize yourself with the documentation. Remove any objects from your workspace. Copy the code to install the package below. (ca. 15 min)}}\label{q5-please-install-the-r-package-epiestim-and-load-it-httpscran.r-project.orgwebpackagesepiestimindex.html-and-familiarize-yourself-with-the-documentation.-remove-any-objects-from-your-workspace.-copy-the-code-to-install-the-package-below.-ca.-15-min}}

~

\hypertarget{q6-install-the-most-used-visualisation-package-in-r-that-is-recommended-by-epiestim}{%
\paragraph{Q6: Install the most used visualisation package in R that is
recommended by
`EpiEstim'}\label{q6-install-the-most-used-visualisation-package-in-r-that-is-recommended-by-epiestim}}

~

\hypertarget{q7-load-and-visualise-flu-incidence-data-contained-in-the-package.-plot-the-figure-and-add-a-caption-to-the-plot.}{%
\paragraph{Q7: Load and visualise flu incidence data contained in the
package. Plot the figure and add a caption to the
plot.}\label{q7-load-and-visualise-flu-incidence-data-contained-in-the-package.-plot-the-figure-and-add-a-caption-to-the-plot.}}

\hypertarget{q8-estimating-time-varying-reproduction-number-rt.}{%
\paragraph{Q8: Estimating time varying reproduction number,
Rt.}\label{q8-estimating-time-varying-reproduction-number-rt.}}

~

We can run estimate\_R on the incidence data to estimate the time
varying reproduction number R. For this, we need to specify i) the time
window(s) over which to estimate Rt and ii) information on the
distribution of the serial interval (SI).

For i), the default behavior is to estimate Rt over weekly sliding
windows (for example: window 1 = 01Jan2020 to 07Jan2020, window 2 =
02Jan2020 to 08Jan2020, window 3 = 03Jan2020 to 09Jan2020, etc). This
can be changed through the config\$t\_start and config\$t\_end arguments
(see below, ``Changing the time windows for estimation''). For ii),
there are several options, specified in the method argument.

The simplest is the parametric\_si method, where you only specify the
mean and standard deviation of the SI.

In this example, we only specify the mean and standard deviation of the
serial interval. In the following example, we use the mean (2.6 days)
and standard deviation (1.5) of the serial interval for flu from
Ferguson et al., Nature, 2005:
\url{https://www.nature.com/articles/nature04017}

Plot the code below and explain each panel. (ca. 10 minutes)

\begin{Shaded}
\begin{Highlighting}[]
\NormalTok{res\_parametric\_si }\OtherTok{\textless{}{-}} \FunctionTok{estimate\_R}\NormalTok{(Flu2009}\SpecialCharTok{$}\NormalTok{incidence, }
\AttributeTok{method=}\StringTok{"parametric\_si"}\NormalTok{,}
\AttributeTok{config =} \FunctionTok{make\_config}\NormalTok{(}\FunctionTok{list}\NormalTok{(}
\AttributeTok{mean\_si =} \FloatTok{2.6}\NormalTok{, }\AttributeTok{std\_si =} \FloatTok{1.5}\NormalTok{)))}
\FunctionTok{plot}\NormalTok{(res\_parametric\_si, }\AttributeTok{legend =} \ConstantTok{FALSE}\NormalTok{)}
\end{Highlighting}
\end{Shaded}

\hypertarget{q9-based-on-the-above-plot-describe-the-epidemiological-dynamics-of-the-outbreak-by-referencing-the-change-in-the-time-varying-reproduction-number-rt.-ca.-10-minutes}{%
\paragraph{Q9: Based on the above plot, describe the epidemiological
dynamics of the outbreak by referencing the change in the time-varying
reproduction number, Rt. (ca. 10
minutes)}\label{q9-based-on-the-above-plot-describe-the-epidemiological-dynamics-of-the-outbreak-by-referencing-the-change-in-the-time-varying-reproduction-number-rt.-ca.-10-minutes}}

~

\hypertarget{q10-estimating-rt-with-a-non-parametric-serial-interval-distribution}{%
\paragraph{Q10: Estimating Rt with a non parametric serial interval
distribution}\label{q10-estimating-rt-with-a-non-parametric-serial-interval-distribution}}

~

If one already has a full distribution of the serial interval, and not
only a mean and standard deviation, this can be fed into estimate\_r as
follows.

Plot serial interval distribution and outputs from this model. Provide a
perspective on what the SI distribution means for estimating
transmission. (ca. 15 minutes)

\begin{Shaded}
\begin{Highlighting}[]
\FunctionTok{plot}\NormalTok{(Flu2009}\SpecialCharTok{$}\NormalTok{si\_distr)}
\NormalTok{res\_non\_parametric\_si }\OtherTok{\textless{}{-}} \FunctionTok{estimate\_R}\NormalTok{(Flu2009}\SpecialCharTok{$}\NormalTok{incidence, }\AttributeTok{method=}\StringTok{"non\_parametric\_si"}\NormalTok{,}
\AttributeTok{config =} \FunctionTok{make\_config}\NormalTok{(}\FunctionTok{list}\NormalTok{(}\AttributeTok{si\_distr =}\NormalTok{ Flu2009}\SpecialCharTok{$}\NormalTok{si\_distr)))}
\FunctionTok{plot}\NormalTok{(res\_non\_parametric\_si, }\StringTok{"R"}\NormalTok{)}
\end{Highlighting}
\end{Shaded}

\hypertarget{q11-estimating-r-accounting-for-uncertainty-on-the-serial-interval-distribution}{%
\paragraph{Q11: Estimating R accounting for uncertainty on the serial
interval
distribution}\label{q11-estimating-r-accounting-for-uncertainty-on-the-serial-interval-distribution}}

~

Sometimes, especially early in outbreaks, the serial interval
distribution can be poorly specified. Therefore estimate\_R also allows
integrating results over various distributions of the serial interval.
To do so, the mean and sd of the serial interval are each drawn from
truncated normal distributions, with parameters specified by the user,
as in the example below:

We choose to draw:

\begin{itemize}
\item
  The mean of the SI in a Normal(2.6, 1), truncated at 1 and 4.2. This
  truncated normal distribution is centred around 2.6, which was the
  mean of the SI distribution in Q8. Here we add uncertainty in the mean
  parameter, but restrict it to lie between 1 and 4.2.
\item
  The sd of the SI in a Normal(1.5, 0.5), truncated at 0.5 and 2.5.
  Similarly, here we add uncertainty to the standard deviation of 1.5
  described in Q8, and limit this to lie between the reasonable range of
  0.5 and 2.5.
\end{itemize}

Provide the outputs from the model and describe the bottom plot in light
of the uncertainty about the SI distribution.

\begin{Shaded}
\begin{Highlighting}[]
\NormalTok{config }\OtherTok{\textless{}{-}} \FunctionTok{make\_config}\NormalTok{(}\FunctionTok{list}\NormalTok{(}\AttributeTok{mean\_si =} \FloatTok{2.6}\NormalTok{, }\AttributeTok{std\_mean\_si =} \DecValTok{1}\NormalTok{,}
\AttributeTok{min\_mean\_si =} \DecValTok{1}\NormalTok{, }\AttributeTok{max\_mean\_si =} \FloatTok{4.2}\NormalTok{,}
\AttributeTok{std\_si =} \FloatTok{1.5}\NormalTok{, }\AttributeTok{std\_std\_si =} \FloatTok{0.5}\NormalTok{,}
\AttributeTok{min\_std\_si =} \FloatTok{0.5}\NormalTok{, }\AttributeTok{max\_std\_si =} \FloatTok{2.5}\NormalTok{))}
\NormalTok{res\_uncertain\_si }\OtherTok{\textless{}{-}} \FunctionTok{estimate\_R}\NormalTok{(Flu2009}\SpecialCharTok{$}\NormalTok{incidence,}
\AttributeTok{method =} \StringTok{"uncertain\_si"}\NormalTok{, }\AttributeTok{config =}\NormalTok{ config)}
\FunctionTok{plot}\NormalTok{(res\_uncertain\_si)}
\end{Highlighting}
\end{Shaded}

\hypertarget{q12-changing-the-time-windows-for-estimation}{%
\paragraph{Q12: Changing the time windows for
estimation}\label{q12-changing-the-time-windows-for-estimation}}

~

The time window can be specified through arguments config\$t\_start and
config\$t\_end. For instance, the default weekly sliding windows can
also be obtained by specifying:

Describe how the estimate of Rt compares to previous estimates (ca. 15
minutes):

\begin{Shaded}
\begin{Highlighting}[]
\NormalTok{T }\OtherTok{\textless{}{-}} \FunctionTok{nrow}\NormalTok{(Flu2009}\SpecialCharTok{$}\NormalTok{incidence)}
\NormalTok{t\_start }\OtherTok{\textless{}{-}} \FunctionTok{seq}\NormalTok{(}\DecValTok{2}\NormalTok{, T}\DecValTok{{-}10}\NormalTok{) }
\NormalTok{t\_end }\OtherTok{\textless{}{-}}\NormalTok{ t\_start }\SpecialCharTok{+} \DecValTok{10} \CommentTok{\# adding 10 to get 11{-}day windows as bounds included in window}
\NormalTok{res\_weekly }\OtherTok{\textless{}{-}} \FunctionTok{estimate\_R}\NormalTok{(Flu2009}\SpecialCharTok{$}\NormalTok{incidence, }
\AttributeTok{method=}\StringTok{"parametric\_si"}\NormalTok{,}
\AttributeTok{config =} \FunctionTok{make\_config}\NormalTok{(}\FunctionTok{list}\NormalTok{(}\AttributeTok{t\_start =}\NormalTok{ t\_start,}
\AttributeTok{t\_end =}\NormalTok{ t\_end,}\AttributeTok{mean\_si =} \FloatTok{2.6}\NormalTok{, }\AttributeTok{std\_si =} \FloatTok{1.5}\NormalTok{)))}
\FunctionTok{plot}\NormalTok{(res\_weekly) }
\end{Highlighting}
\end{Shaded}

\hypertarget{q13-specifying-imported-cases}{%
\paragraph{Q13: Specifying imported
cases}\label{q13-specifying-imported-cases}}

~

All of the above assumes that all cases are linked by local transmission
(eg- within country). Sometimes you may have information (from field
epidemiological investigations for instance) indicating that some cases
are in fact imported (from another location (eg internationally), or
from an animal reservoir). See some more details on the estimation in
this publication:
\url{https://www.sciencedirect.com/science/article/pii/S1755436519300350?via\%3Dihub}.
We allow to include such information, when available, as illustrated in
the example below: generating fake information on our cases:

Compare estimates of Rt for this model compared to earlier models that
do not consider imported cases. What is the impact on local Rt? (ca. 15
minutes)

\begin{Shaded}
\begin{Highlighting}[]
\NormalTok{dates\_onset }\OtherTok{\textless{}{-}}\NormalTok{ Flu2009}\SpecialCharTok{$}\NormalTok{incidence}\SpecialCharTok{$}\NormalTok{dates[}\FunctionTok{unlist}\NormalTok{(}\FunctionTok{lapply}\NormalTok{(}\DecValTok{1}\SpecialCharTok{:}\FunctionTok{nrow}\NormalTok{(Flu2009}\SpecialCharTok{$}\NormalTok{incidence), }
  \ControlFlowTok{function}\NormalTok{(i) }
  \FunctionTok{rep}\NormalTok{(i, Flu2009}\SpecialCharTok{$}\NormalTok{incidence}\SpecialCharTok{$}\NormalTok{I[i])))]}

\NormalTok{location }\OtherTok{\textless{}{-}} \FunctionTok{sample}\NormalTok{(}\FunctionTok{c}\NormalTok{(}\StringTok{"local"}\NormalTok{,}\StringTok{"imported"}\NormalTok{), }\FunctionTok{length}\NormalTok{(dates\_onset), }\AttributeTok{replace=}\ConstantTok{TRUE}\NormalTok{)}

\NormalTok{location[}\DecValTok{1}\NormalTok{] }\OtherTok{\textless{}{-}} \StringTok{"imported"} \CommentTok{\# forcing the first case to be imported}

\DocumentationTok{\#\# get incidence per group (location)}
\NormalTok{incid }\OtherTok{\textless{}{-}} \FunctionTok{incidence}\NormalTok{(dates\_onset, }\AttributeTok{groups =}\NormalTok{ location)}

\FunctionTok{plot}\NormalTok{(incid)}

\DocumentationTok{\#\# Estimate R with assumptions on serial interval:}
\NormalTok{res\_with\_imports }\OtherTok{\textless{}{-}} \FunctionTok{estimate\_R}\NormalTok{(incid, }\AttributeTok{method =} \StringTok{"parametric\_si"}\NormalTok{,}
\AttributeTok{config =} \FunctionTok{make\_config}\NormalTok{(}\FunctionTok{list}\NormalTok{(}\AttributeTok{mean\_si =} \FloatTok{2.6}\NormalTok{, }\AttributeTok{std\_si =} \FloatTok{1.5}\NormalTok{)))}

\CommentTok{\# Default config will estimate R on weekly sliding windows.}
\CommentTok{\# To change this change the t\_start and t\_end arguments.}
\FunctionTok{plot}\NormalTok{(res\_with\_imports, }\AttributeTok{add\_imported\_cases=}\ConstantTok{TRUE}\NormalTok{)}
\end{Highlighting}
\end{Shaded}

\hypertarget{q14-name-other-sources-of-uncertainty-in-the-rt-estimation}{%
\subsubsection{Q14: Name other sources of uncertainty in the Rt
estimation?}\label{q14-name-other-sources-of-uncertainty-in-the-rt-estimation}}

params \textless- data.frame( Parameter = c(``Transmission Rate (β)'',
``Recovery Rate (γ)'', ``Incubation Rate (ω)'', ``Force of Infection
(𝛌)'', ``Migration rate (ẟ)'', ``N'', ``Evolutionary Rate''), Value =
c(``0.32'', ``0.157 per day'', ``0.25 per day'', ``Estimated'', ``0,
after 30 days increases to 0.01 per day'', ``3,000 (per deme)'',
``1.1x10-3''), Source =
c(``\url{https://www.sciencedirect.com/science/article/pii/S2405844021009154}'',
``\url{https://www.fhi.no/en/id/infectious-diseases/coronavirus/coronavirus-modelling-at-the-niph-fhi/}'',
``\url{https://www.fhi.no/en/id/infectious-diseases/coronavirus/coronavirus-modelling-at-the-niph-fhi/}'',
````,''``,''``,''\url{https://academic.oup.com/ve/article/6/2/veaa061/5894560}``)
)

params \%\textgreater\% kable(``html'', escape = F, align = ``l'',
caption = ``Model Parameters'') \%\textgreater\%
kable\_styling(bootstrap\_options = c(``striped'', ``hover'',
``condensed''))

\end{document}
